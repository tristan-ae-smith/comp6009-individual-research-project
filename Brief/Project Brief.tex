\documentclass[12pt]{article}
%\usepackage{a4wide}
%\usepackage[small,compact]{titlesec}
\usepackage[parfill]{parskip}
\date{}
%\raggedbottom
\thispagestyle{empty}
\setlength{\topmargin}{-0.2in}
\setlength{\textheight}{8.7in} 
\begin{document}
\begin{center}
{\LARGE Procedural Content Generation\\[0.25cm] for Computer Games}


Thomas Smith (\texttt{taes1g09@ecs.soton.ac.uk})\linebreak
{\bf Supervisor:} Rikki Prince (\texttt{rfp@ecs.soton.ac.uk})
\end{center}

{\bf Problem:} Procedural Content Generation (PCG) is the name applied to the growing field of techniques that attempt to provide content for modern games through the use of algorithmic generators rather than manual construction. As AAA games get larger and indie studios get smaller, all varieties of computer game developer have an incentive to automate the production of material of all kinds for their games. A large number of specialised techniques exist, each tailored to generating a particular kind of content, from rulesets to textures to enemy or object behaviour. Increasingly, concepts from the field of AI are being used to improve the quality and relevance of the generated content, whether that be through intelligently breeding possibilities from a population of solutions or applying machine learning to create a map between player behaviour and optimal content. Each genre of game and each kind of generated content has a range of potential techniques that may be useful for its production, and typically the PCG techniques used in games are individual bespoke applications of these various methods as each game requires different content and there are not yet standard approaches to integrating multiple kinds of content generation.


%In modern computer game development, content production accounts for a large proportion of the initial (and in some cases, ongoing) outlay. As both budgets and in-game worlds get larger, there is increasing demand to offload some of these production efforts to automated systems. The concept of procedural content generation (PCG) has been around for some time, and it has been used in many successful games. When content is generated manually or algorithmically during the design phase of a game, it can only be created according to the designer's expectations of the players' needs. By instead generating content during the execution of the game, and using information about the player(s) as one of the system's inputs, PCG systems should be able to produce more dynamic experiences that can be far more tailored to enhance individual player's experiences than anything manually created.

{\bf Goals:} Many existing papers on PCG deal with a single aspect or particular technique. A smaller number consider a range of previous techniques for the generation of a particular range of game content, such as terrain or cities, and draw comparisons on their relative merits. The goal of this project will be to attempt a high-level classification of a large range of techniques used for many different types of content, and attempt to discern common aspects that can help to promote a more unified, standard approach to content generation for games. The final output will consist of a detailed report, an A1 poster and accompanying presentation, and hopefully a short paper for submission to the Fourth Workshop on Procedural Content Generation in Games at FDG2013.

{\bf Scope:} In order to attempt to ensure that the project goals remain achievable, the scope should be restricted to only recognised and established PCG techniques with obvious application to games. It may be useful to consider related areas for comparison, such as player-driven content generation or procedural PCG generators, but the focus should remain limited to applicable and useful methods in order to facilitate classification and comparison.

\end{document}

lots of genres
lots of genre-specific solutions
range of content - levels, music, textures, rules, weapons, storyline? meta
range of methods

goals - necessarily highlevel