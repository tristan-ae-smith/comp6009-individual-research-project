\documentclass{acm_proc_article-sp}
\usepackage{graphicx}
\usepackage{multirow}

\begin{document}

\title{Procedural Content Generation for Computer Games}
% \subtitle{A survey of techniques used for procedural content generation for computer games, classified by beneficiary.}

\numberofauthors{1}
\author{
\alignauthor
Thomas Smith\\
       \affaddr{Electronics and Computer Science}\\
       \affaddr{University of Southampton}\\
       \email{taes1g09@ecs.soton.ac.uk}
}

\maketitle
\begin{abstract}
The modern computer games industry makes use of a wide variety of procedural content generation (PCG) techniques that serve a number of purposes during both the development and execution of a game. This paper provides a high-level classification of a range of techniques used for different types of content, and attempts to discern common aspects and transferable approaches that can help to promote a more unified, standard approach to content generation for games. 


% Improve variety, allow developers to populate large areas, allow customisation to player preferences, reduce assets, allow control/tweaking, reduce labor costs
% algorithmic generators rather than manual construction. 
% incentive to automate the production of material
% specialised techniques, each tailored to a particular kind of content
% concepts from AI used to improve the quality and relevance of the generated content
% intelligently breeding possibilities from a population of solutions or applying machine learning to create a map between player behaviour and optimal content. 
% Each genre of game and each kind of generated content has a range of potential techniques that may be useful for its production, 
% each game requires different content
% there are not yet standard approaches to integrating multiple kinds of content generation.



% Hundreds of millions of people play computer games every day. For them, game content–from 3D objects to abstract puzzles– plays a major entertainment role. Manual labor has so far ensured that the quality and quantity of game content matched the demands of the playing community, but is facing new scalability challenges due to the exponential growth over the last decade of both the gamer population and the production costs. Procedural Content Generation for Games (PCG-G) may address these challenges by automating, or aiding in, game content generation. PCG-G is difficult, since the generator has to create the content, satisfy constraints imposed by the artist, and return interesting instances for gamers. Despite a large body of research focusing on PCG-G, particularly over the past decade, ours is the first comprehensive survey of the field of PCG-G. We first introduce a comprehensive, six-layered taxonomy of game content: bits, space, systems, scenarios, design, and derived. Second, we survey the methods used across the whole field of PCG-G from a large research body. Third, we map PCG-G methods to game content layers; it turns out that many of the methods used to generate game content from one layer can be used to generate content from another. We also survey the use of methods in practice, that is, in commercial or prototype games. Fourth and last, we discuss several directions for future research in PCG-G, which we believe deserve close attention in the near future.
\end{abstract}

\category{K.8.0}{Personal Computing}{General}[Games]

\terms{Algorithms, Design, Standardization}

\keywords{Procedural generation, game development}

\section{Introduction}

Though procedural content generation (PCG) techniques have been used since some of the earliest computer games \cite{elite}, they are now becoming increasing relevant as big-budget commercial games become larger and more detailed and indie games teams become smaller. PCG techniques can easily amplify production efforts by automating asset generation or augmenting manual content production, and they can be a powerful tool to allow developers to create and populate varied and believable playspaces. They can also play a role in customising each player's experience to improve engagement and entertainment. 

However, the wide range of possible applications for PCG has lead to an extraordinary variety of approaches from both academia and industry. Despite the extensive research undertaken over three decades, there are still no general purpose procedural game content generators, or even standard approaches to common requirements. Current procedural content generators are typically bespoke approaches used to provide specific content for a particular application. These approaches are not usually portable to other domains. This paper shows however that there are a number of particular algorithms often used as starting-points for these solutions, and these algorithms could lead to standard approaches.

\subsection{What is PCG?}
% One of the problems facing serious researchers into PCG is that the field itself is impossible to define. One might as well ask the same question about birds - what are birds? We just don't know.
One of the problems facing researchers into PCG is the difficulty in defining the field itself.
Hendrikx et al. describe it as ``the application of computers to generate game content, distinguish interesting instances among the ones generated, and select entertaining instances on behalf of the players.'' \cite{hendrikx2012procedural}, however this definition does not cover simple unevaluated or human-evaluated content such as deterministic methods, or tools used by artists during development, and does not suggest how entertainment may be evaluated. Another definition is provided by Togelius et al.: ``[PCG is] the algorithmical creation of game content with limited or indirect user input.'' \cite{togeliusPCGdef}. They also note that this definition does not specify the presence or absence of adaptivity or randomness, as PCG methods might be both, either or none. This definition hinges on what is understood by the phrase ``game content'', as many modern approaches to PCG could be said to generate more general game `experiences' than the traditionally-understood `content'.

\subsection{Online vs. Offline}
An easily available classifier for PCG approaches is whether the content generation is performed \textit{offline}, during the development of the game or as it loads, or \textit{online} as a result of player actions \cite{togelius2011search}. Originally, PCG algorithms ran offline due to the processing load they incurred, and the results can be baked into the game's data and shipped. This saves artists' and designers' labour during development, while still allowing fine-tuned control over the final output. Alternatively, content can be generated as a game or level loads - meaning that the result can vary between different players and different playthroughs, and reducing the storage needed for game content. Recent progress has led to faster algorithms that are capable of running during gameplay and generating new content on-the-fly; e.g. to generate more content whenever the boundaries of existing content are approached. The development of online PCG is more demanding than offline versions, as available processing resources are more constrained, and algorithms must have predictable runtimes, however it brings advantages via providing the option for PCG to adapt to varying game states.

\subsection{Methodology}
Given the broad range of procedural content techniques, there are a number of different categorisations possible. Hendrikx et al.\ provide a taxonomy that describes content in terms of how it is formed - a heirarchy is presented in which upper classes may be built using elements from lower ones (Table~\ref{tab:hendrikx}).
This review organises approaches first by the primary user of each method - Artists, Designers and finally Users. This leads to a natural progression from primarily-offline approaches to more recent online algorithms. Each section details a number of the main purposes that PCG is used for within that category, along with examples and academic literature in that area. Since in some areas there are an abundance of similar methods references to surveys have been preferred where possible, with papers on specific techniques occasionally used if they also provide a broader overview of their area, or illustrate an uncommon approach.

\begin{table}[!ht]
  \begin{tabular}{r|l}
  Content class		& Content type examples\\
  \hline 
   \\[-2ex]
  Derived Content	& News and Broadcasts, Leaderboards\\ [.5ex]
  Game Design		& System Design, World Design\\ [.5ex]
  Game Scenarios	& Puzzles, Storyboards, Story, Levels\\ [.5ex]
  \multirow{2}{*}{Game Systems}
  					& Ecosystems, Road Networks,\\
  					& Urban Environments, Entity Behaviour\\ [.5ex]
  Game Space		& Indoor/Outdoor Maps, Bodies of Water\\ [.5ex]
  \multirow{2}{*}{Game Bits}
  					& Textures, Sound, Vegetation, Buildings,\\
  					& Behaviour, Fire, Water, Stone \& Clouds\\ [-2ex]
  \end{tabular}
  \caption{Taxonomy provided by Hendrikx et al. \cite{hendrikx2012procedural}}
  \label{tab:hendrikx}
\end{table}

\section{Artists}
%methods and results
Artists have benefited greatly from the development of PCG techniques, as they can work more efficiently by automating generation of many content types, and can produce a wider range of outputs by automating the creation of variants. A number of techniques have arisen to generate realistic non-repeating textures, assemble a range of varied models from component parts, intelligently animate character skeletons, produce realistic and varied visual effects, and even supply cue-dependent responsive music. Many of these methods are not unique to the field of computer games, and have been adopted to produce computer-generated graphics for all kinds of media, from animated movies to photo-realistic CGI backgrounds in advertisements and print media. Part of the reason for these techniques' popularity in other fields is the fact that they are about automating the construction of essentially static resources - all either run offline during development, or have no effect on gameplay. 

\subsection{Content types}
Artists make use of PCG to streamline the workload involved in producing a wide range of content types, and common approaches have emerged for some of the most typical use cases, as detailed below:

\subsubsection{Textures}
Some of the most common approaches for procedural texture construction are pseudo-random number generators such as Perlin noise, algorithmic approaches for specific repeating effects, and image filtering for pattern-based textures.
Games in particular have a long history of using Perlin noise and similar effects for procedural material textures. Originally developed by Ken Perlin for the motion picture industry, it provides natural appearing textures via a deterministic process. With careful tuning of parameters, it may be used for a range of applications, from clouds to stone to wood to marble \cite{perlin2002improving}.
In contrast, a number of other algorithmic approaches have been developed for generating repeating textures such as the windows on high-rise tower blocks, segments of steel plating or stretches of road. Some of these have been adapted to provide non-repeating high-resolution texturing for rolling natural landscapes and backdrops via image filtering \cite{patternTextures}.
While procedural texture generation is most usually undertaken offline for artists during development, online algorithms exist that allow comparatively unskilled players to automatically generate suitable textures with limited interaction \cite{DeBry:2007:PPT:1278780.1278878}.

\subsubsection{Models}
% visual variety, procedural construction
% fill space - speedtree\cite{speedtree}, procedural cities \cite{carli2011survey}
% Borderland's guns TODO
% Grammars - SpeedRock \cite{dart2011speedrock}
% Speedtree
Creating a modular `kit' of compatible reusable model components is a common technique to reduce artist workloads and increase available variety, however with the addition of procedural arrangement algorithms large parts of the process can be automated and hundreds of unique `individuals' created with comparatively little effort. Such approaches often define generative grammars such as L-systems to ensure individual validity, and include refinement passes to improve the quality of generated output \cite{dart2011speedrock}. One popular example of these systems is Speedtree - a commercial middleware product capable of generating many varieties of realistic foliage, from single bushes to entire forests \cite{speedtree}.

\subsubsection{Animation}
% Rather than create animations for all possibilities (counterexample - assassins creed 3 had <number> of distinct animations)
% Respond to conditions that weren't known at design time: user content generation - Spore\cite{Spore}, allows characters to react to a vast range of physical conditions - Jedi Unleashed force push, Emotion engine.
The production of skeletal animations for modern 3D games is a time-consuming process -- modern games have thousands of animations, and often use PCG techniques in order to automate parts of the work involved in ensuring characters behave realistically. Given known finish and start states for key animations provided by hand or motion capture, a combination of interpolation and physics-aware PCG methods can be used to generate all of the transitions between pairs of animation states \cite{ac3}.
In addition to the offline process, some games also use an online PCG system that intelligently blends animations and ragdoll behaviours in order provide entities with fluid and realistic reactions to unexpected situations. One of the best known such systems is Euphoria, an animation engine that uses a skeletomuscular simulation to generate realistic animation of characters at runtime \cite{motion2006euphoria}.
A number of approaches also use PCG to develop animations for contexts and conditions that could not have been known at design time, typically in cases where there is a degree of user-driven content generation \cite{horswill2009lightweight,Perlin:2008:FVP:1401843.1401854}.

\subsubsection{Effects}
% Procedural generation of environmental effects such as fire, water, smoke and clouds. Provides believable variety (starter point for citations in \cite{hendrikx2012procedural})
% Particles
% Generation of effects representing spells and weapons allows customisation based upon the properties of the ability \cite{particles}
% Procedural rendering effects - allows graphical styles that are radically distinct from traditional photo-realistic techniques \cite{kowalski1999art}
In contrast to the typically-offline approaches presented so far, many PCG techniques exist that are designed to display interesting visual effects at runtime. The artist's role when using these systems will be to tune parameters of the PCG system in order to produce the desired effect, such as fire, water, smoke or clouds. The use of PCG to generate and tune such effects provides believable variety, and can give an additional degree of control in comparison to pure particle systems -- which in themselves are a form of stochastic procedural modelling \cite{reeves1983particle}. By specifying the generation of such effects at runtime, it is possible to create effects that respond dynamically to their contexts, or evolve based upon external factors \cite{particles}.
Another class of effects possible via PCG is the procedural manipulation of the scene during rendering in order to create graphical styles that are radically different from traditional photo-realistic techniques -- such as the tufted truffala trees of Dr. Seuss \cite{kowalski1999art}.


\subsubsection{Music}
As with other content types discussed thus far, the application of PCG techniques to existing approaches to producing music for videogames allows a vast increase in the variety of content available for the same initial investment. Music in games is often cued to player actions, however storage available for individual themes is finite, and so melodies may become repetitive if the same track is triggered many times throughout the course of a game \cite{collins2009introduction}. One approach to solving the occurrences of repetitive music is the use of online transformational algorithms that are able to restructure tracks or alter overall pitch or tempo at runtime, often in response to specific aspects of the context in which the track is played. Another more complex technique involves building bespoke compositions for individual situations from a selection of precomposed components. By providing a generative grammar and a library of smaller tune fragments tagged to fit specific themes, sound designers are able to produce far greater expressive variety and avoid the issues caused by limited musical palette. This approach also lends itself to greater integration between the musical soundscape and the players actions, as the music is able to reflect several aspects of the context simultaneously \cite{6266725}.

\subsection{Approaches}
\label{sec:artist-approaches}
Though there are a wide range of techniques, including many unique cases that defy categorisation, it can be seen that the PCG systems used by artists fall broadly into two principal classes. On the one hand are the approaches that are purely algorithmic, as with the Perlin-generated textures \cite{perlin2002improving} and the particle-based effects \cite{reeves1983particle}. These achieve efficiencies in both workflow and storage requirements by specifying a reusable generation system and then only the sets of parameters needed to generate the specific desired content. Contrast to these the \textit{component-assembly} approaches that begin with a kernel of hand-generated components, and procedurally assemble these modules and refine the output. Examples of this class are the pattern-based textures \cite{patternTextures}, grammar-based model constructors \cite{dart2011speedrock} and blended music \cite{6266725}. These approaches have the effect of magnifying the small pool of original content into a range of varied outputs.

Despite the differences in implementation, the common underlying intention of the two approaches leads to a number of similarities between them. None of the methods described have any automatic evaluation component, as it is assumed that human evaluation by the artists using the techniques will always be possible, and have superior results. The majority of them are designed to be used offline during development -- where this is not the case, they are either tuned during development and have no impact on gameplay, as is the case with procedural effects and music, or where they are designed for user interaction they are typically made as foolproof as possible, and have minimal effect on gameplay.

%comparison and evaluation of approaches
\subsection{Benefits} %TODO talk about costs
Two of the main benefits associated with artists' use of PCG techniques are the reduction in labour required to make a large variety of variations on a them, and the reduced storage requirement to represent this variance. By replacing hundreds of textures, models or animations with a set of components and a system for assembling them into content, the size of necessary assets can be dramatically reduced. `.kkrieger', a recent demoscene FPS game, uses entirely procedurally generated assets and requires under 100KB of storage (.theprodukkt, 2004). This is becoming increasingly relevant as games become larger and the status quo shifts from physical media to digital distribution - smaller games means faster delivery and less bandwidth costs. %Everything old is new again
% Download sizes - procedural variation - Borderland's enemies
% When producing the raw content that goes into a game, procedural generation techniques can provide a more efficient method or greater variety than building everything by hand. 

\subsection{Future Work}
Though PCG is currently effectively used by artists both in the games industry and other digital media fields, there are a number of known issues and other ways in which existing approaches may be refined or improved upon. Amongst the algorithmic approaches, it can be difficult for artists to correct a generated texture that is almost but not quite fit for purpose. Exposed algorithm parameters do not usually provide the fine degree of control that might be necessary, and while manual correction of the texture is always possible during development, such edits would be lost if the texture is regenerated or needs to be generated during runtime.
Another issue for some content types is the challenge of producing interesting, unique content. For many content types such as textures and foliage, a range of slightly varied repetitions on a theme are precisely what is needed. However for in-game audio the goal is to combine the available fragments in fresh ways to avoid repetition -- the difficulties in doing so effectively are the main reason transformational approaches are currently preferred \cite{collins2009introduction}.
Finally, a opportunity for the field in general is the standardisation of approaches to generative grammar PCG systems. Though SpeedTree provides a commercially-available middleware solution, it is restricted to the generation of foliage only, and other more typical solutions in this area are bespoke approaches -- highly dependant upon the nature of the problem being solved, the flavour of generative grammar chosen, and the domain-specific grammar itself.



\section{Designers}
In contrast to the procedurally generated content developed by artists, which typically has no direct effect on gameplay, the PCG techniques used by designers are generally oriented towards producing content that will have direct in-game effects during runtime. Though PCG techniques can be used to generate swathes of content in a more efficient manner than was previously possible, it also brings other, more valuable benefits to designers. The addition of a stochastic component to most approaches means that output may vary dramatically between one execution and the next, providing greater replayability. However, this comes at the cost of being unable to fully evaluate the output of the generator during development. Unless the system is simple enough that its entire expressive range is known to be usable, there is a risk of \textit{catastrophic failure} \cite{togelius2011search}, where potentially uncompletable content is generated. To combat this, online PCG approaches that can impact gameplay typically follow a `generate-and-test' pattern, whereby each generator includes an evaluator for valid content, and output that fails evaluation is rejected and regenerated.
%methods and results

\subsection{Content Types}
Designers make use of PCG to provide varied and interesting play spaces in a multitude of ways. Sample PCG approaches for some of the most common aspects of playspaces are detailed below:

\subsubsection{Enclosed Environments}
Many games incorporate some manner of enclosed environments where the gameplay is highly influenced by the topology of the playspace -- these could be anything from caves to pinball tables to temples to racetracks. Whether 2D or 3D, it is generally necessary that the environment fulfils a certain set of minimum constraints so that it is fit for purpose -- often including the requirement that the final output contain at least one entrance and exit, and that the space between them is ultimately traversable. The specific method used to generate an enclosed environment and then evaluate the output is typically highly domain dependent, though some general approaches do exist. The component-assembly method described in section~\ref{sec:artist-approaches} is a common starting-point, often combined with advanced route creation in order to ensure an interesting and feasible path exists through the level \cite{smith2011launchpad}. Variants exist that offer `mixed-initiative' development -- the ability for human designers to intervene and modify the developing level at any point without disrupting the PCG algorithm \cite{mawhorter2010procedural}. Another approach suitable to more natural-looking enclosed environments is the use of cellular automata to erode randomly-seeded maps in an iterative fashion, generating extensive, tunable cave systems \cite{johnson2010cellular}. For simpler applications, A$^*$ search is a basic traversability evaluator, though domain knowledge may allow more efficient approaches in many cases.

\subsubsection{Open Environments}
Open environments differ from enclosed environments primarily in that they constrain gameplay to a lesser degree, and in turn are less constrained by it. Open environments are generally considerably more sparse and rarely unsatisfiable in any sense -- rather, in this context PCG techniques are primarily used to provide believable variety for the form and content of the playspace, and must be of sufficient quality to do so. A wide range of techniques exist to generate many kinds of terrain, and existing noise and network creation techniques may be used to specify land cover features such as woodland, roads and rivers \cite{raffe2012survey,perlin2002improving}. A number of similar techniques have been developed to populate urban environments with realistic transport infrastructure and buildings \cite{carli2011survey}. Again, approaches have been developed that allow human designers to influence the generation of content in support of particular goals, either through the introduction of \textit{semantic constraints} that express localised intent such as the intended location of a valley or lake \cite{smelik2011semantic}, or by introducing broader \textit{objectives} that specify purpose for the generated environment, such as a mountainous region full of cover, or an easily traversable plain \cite{togelius2010towards}.

\newpage
\subsubsection{Entity Behaviours}
\label{sec:GAR1}
In addition to the generation of more concrete content types, PCG techniques may be used to manipulate the behaviours of entities such as NPCs, vehicles or weapons within the game. This can be done in real time in order to progressively modify a character's responses to the player, as in the interactive drama Fa{\c{c}}ade \cite{mateas2007writing}, or it may be integrated into entity development in order to provide enemy and ally characters with a range of believable fighting styles, as in the FPS Killzone \cite{straatman2005killzone}.
Some modern games also procedurally generate the behaviour and appearance of non-character entities, particularly weapons. Both Borderlands (Gearbox Software, 2009) and Galactic Arms Race (GAR, Evolutionary Games, 2010) make use of runtime PCG techniques in order to provide an almost limitless variety of weapons with a range of effects, potencies and appearances. Entirely different approaches are used however: Borderlands uses the component-assembly method to provide visuals for the weapons, and a stochastic generator to specify their behaviours, while GAR uses particle system visuals and an evolutionary algorithm to generate behaviours.

\subsection{Approaches}
%comparison and evaluation of approaches
There are a wide range of approaches to PCG used by designers for an equally wide range of purposes -- including many highly domain-specific solutions not covered here. However, unless the approach is a simple \textit{constructive} PCG technique that is guaranteed never to produce broken content, all of these approaches incorporate some kind of evaluation method to ensure that each piece of content it produces is viable in some sense. Some of the more common approaches include the use of cellular automata, generative grammars for top-down planning or the component-assembly model seen in section~\ref{sec:artist-approaches}, and the use of evolutionary algorithms for content generation.
However, rather than specific algorithmic approaches, two main techniques are observed from many of the techniques studied.

\subsubsection{Search-Based PCG}
An extension of the generate-and-test approach, search-based PCG evaluates candidate outputs according to some domain-specific fitness function, and assigns each some appropriate fitness score. The generation of further candidate content is then influenced by the fitness of existing content, with the aim of ensuring each iteration has increasing fitness. With a carefully chosen fitness function this means not only that all output from a search-based PCG system will be valid ( traversable / interesting ), but also that successive uses of the generator improves the general quality of generated content \cite{togelius2011search}.

\subsubsection{Mixed-Initiative Creation}
Another common feature of some of the PCG approaches surveyed is the provision for human designers to directly influence the generation of content and therefore the ultimate output of the system. This is known as mixed-initiative creation as it allows the provision of hand-generated set pieces \cite{smith2010tanagra}, level features \cite{mawhorter2010procedural} or constraints \cite{smelik2011semantic} by a human developer, and then uses the PCG system to fill in the blanks and weave the provided parts into a coherent whole. These systems can also provide live editing capabilities after generation, and warn the developer if the resulting level fails any of a number of satisfiability constraints.
\newpage
\subsection{Benefits}
The use of PCG techniques by designers in industry is typically on an as-needed basis. Though entire games could be constructed around a single robust PCG mechanic, in the more general case there will be one or two facets of a game that receive specific advantages due to the use of PCG techniques. Most commonly, these benefits will be one or more of the following:
% speed up work - large, believable diverse playspaces\\


\subsubsection{Content Scale}
PCG techniques allow designers to generate and then populate vast game spaces with a high level of both detail and variety. Middleware packages such as Speedtree\cite{speedtree} allow entire forests to be generated and customised, and similar approaches exist for generating believable cities and landscapes \cite{carli2011survey}. The use of online PCG systems for environment generation allows the creation of `borderless' playspaces, that continually generate new content on-the-fly whenever a boundary is approached \cite{mario}.
% ``A Survey of Procedural Terrain Generation Techniques using Evolutionary Algorithms'' \cite{raffe2012survey}

\subsubsection{Replay Value}
The use of stochastic PCG approaches for environment generation means that playspaces may be different for each player, and for each player's playthrough of a game. The variety ensures that content is fresh each time, and minimises the effects of repetition -- players are not able to memorise the precise route through each dungeon and locations of treasures, and so there is a sense of discovery and exploration each time \cite{hendrikx2012procedural}. Some games make the original seed value for each world available to share, allowing parallel exploration.

\subsubsection{Challenge Adjustment}
\label{sec:DDA}
On of the main factors that affects players' enjoyment of games is their ability to remain in a state of `flow' -- the sensation that their abilities are matched to the challenge provided by the game \cite{flow}.
In order to cater to the wide range in ability of players that each game may attract, many games have previously offered the option of customising the challenge experienced to one of a number of pre-set `difficulty levels'. 
% <Read through 3yp for citations on difficulties with self-assessment and mutability>. 
Rather than specify the degree of challenge at design time, it is possible to adaptively generate or alter aspects of the game \cite{lopes2011adaptivity} in order to match the observed ability of the player \cite{ResE5}.

\subsection{Future work}
One of the issues with the `generate-and-test' pattern (and with search-based PCG) is that if the evaluation function is inefficient or incorrect it can consume significant processing time during runtime, or mistakenly validate and emit untraversable content \cite{hendrikx2012procedural}. The search for efficient and effective evaluation algorithms is an ongoing one, as many forms of content are easier to procedurally generate than to procedurally animate.

% \newpage
\section{Users}
%methods and results
With the advent of PCG algorithms efficient enough to be executed online during gameplay, a new form of input for adaptive PCG has become available: the actions of the player may be used to influence the output of the PCG system.
The ability for PCG content to respond in some way to the user helps to close the gap between designer and player, and allows modification of the game state tailored to individual information about the player \cite{sorenson2010towards}.

\subsection{Content Types}
Users may influence generated content in a number of ways. Where the purpose of player-adaptive PCG is to tailor the experience to some aspect of the player, then the user's influence is likely to be indirect and ideally go unnoticed \cite{citeulike:7364450}. Alternatively, directly player-adaptive PCG systems may be used as a form of user-driven content generation, in order to increase the player's perceived agency within the world \cite{bartle2004designing}.
%TODO: say why either of these might not be PCG

\subsubsection{Experience}
Many games have used the availability of player information during runtime in order to procedurally generate experiences that are tailored specifically to certain aspects of the current player. The dynamic difficulty adjustments mentioned in section~\ref{sec:DDA} are a basic example of this process, however the concept supports many other interesting opportunities, especially when combined with decision-making techniques from the field of AI. 

For example Valve's AI Director \cite{valve}, as seen in co-operative zombie shooters `Left 4 Dead' (Valve Software, 2008) and `Left 4 Dead 2' (Valve Software, 2009), is an artificial intelligence system that maintains a model of each player's \textit{emotional intensity} and then uses PCG techniques to effect wide-ranging changes across each scenario in order to provide a dynamic challenge. It is able to affect the spawn rates and location of enemies, the lighting, the weather and even the layout of the level in order to heighten tension and maintain challenge.

Another system that generates content influenced by a model of the player is Bethesda's Radiant Storytelling system \cite{radiant} as used in `Skyrim' (Bethesda Softworks, 2011). Skyrim is a vast open-world game that uses exploration as a primary gameplay mechanic, and the Radiant Storytelling system maintains a model of interesting locations that the player has not yet visited. Whenever the player interacts with any designated Radiant Questgiver NPC, the system procedurally generates a new mission designed to draw the player to an unvisited location. Depending on the nature of the template selected for the quest, this can involve replacing the inhabitants of the target location with some hostile creature type, or ensuring that a particular named NPC or item is spawned at pre-designated spots in that location. The Radiant system is designed to provide a natural and engaging flow between locations in the game world

A more academic investigation of these techniques is provided by Nitsche et al. in their game Charbitat \cite{charbitat}. Charbitat is an open-world exploration game in which the world itself is generated in response to the player's actions. The tiles of the world each have elemental affinities to certain degrees, and whenever a new tile is generated its nature is determined by a combination of the natures of the adjoining tiles and the elemental nature of the player character, which varies according to the players actions. Thus players may deliberately but indirectly influence the nature of the world that is generated in response to their actions.

\newpage
\label{sec:GAR2}
Finally, the commercially-successful academic experiment GAR (see section~\ref{sec:GAR1} and \cite{garmsr}) uses an evolutionary algorithm to breed new varieties of weapons system. The breeding system keeps track of which weapons are most used by the player, and accords these individuals preferential selection, resulting in a population which tailors itself over time towards the weapon types most favoured by the player - with occasional mutants injected to prevent homogenisation.

\subsubsection{Agency}
Typically, players have no direct control over adaptive generation, as player influence on PCG systems is generally indirect \cite{hamlet}.
However, in some games that make heavy use of procedural generation, giving the player some degree of direct control over the generation process can improve their perception of their agency within the world \cite{bartle2004designing}.

For example, GAR introduced the `Weapons Lab' -- a portion of the game where players may spend in-game resources to customise their procedurally-generated weapons. Each weapon is represented by an evolved network, and players are given the ability to tweak individual weights and observe the effects on the performance of each shot. This gives players a degree of direct control over systems within the game that were previously only indirectly accessible, increasing the depth of gameplay available \cite{Hastings:2010:IGE:1814256.1814264}.

Another example of a commercial game exposing PCG systems is `Spore' (Electronic Arts, 2008), in which players are given access to a range of simple editors for creatures, vehicles, structures and spaceships. All existing entities of these kinds are already procedurally generated by the game, and the editors allow players to specify further personal creations, which are then procedurally generated, textured and animated \cite{DeBry:2007:PPT:1278780.1278878,Perlin:2008:FVP:1401843.1401854}.

The concept of agency via influence on PCG systems is explicitly investigated by `Rathenn' (Gillian Smith et al., 2011) an experimental PCG-based 2D platformer. As players reach the end of each segment they are faced with a choice of ladders that may be used to ascend to the next segment. The ladders are colour-coded according to the effects they each have on three adaptive parameters of the PCG system: the frequency of enemies vs. gaps; the frequency of jumps vs. waits; and the speed and agility of the avatar. These choices allow the player to explore the expressive range of the generator whilst exploring the resulting generated space \cite{smith2011pcg}.
% Inside a star-filled sky citation in \cite{smith2011pcg}

\subsection{Approaches}
%comparison and evaluation of approaches
The approaches to player-adapted PCG listed here use a wide range of techniques to demonstrate the player's influence on the virtual world. The only common feature of all these methods is that they form a conceptual model of some relevant aspect of the player, and then use that information about the player as an input parameter for an adaptive generator. In some cases (Skyrim, Left 4 Dead, Charbitat) this involves intermediate processing and decision-making by an AI system, while in others (Spore, GAR) this leads to a form of user-driven content generation. In all cases, players are presented with content or experiences that are uniquely their own, as a result of decisions that they have made.
% The Big AI\\
% User-driven content generation

\subsection{Benefits}
In games where player-adapted PCG is used to generate content in response to player actions, users receive a more engaging and immersive experience that is unique to them. Content can be tailored specifically to their preferences and abilities, and it can be possible for them to directly customise certain kinds of generated content, increasing perception of ownership and agency \cite{Hastings:2010:IGE:1814256.1814264}.

\subsection{Future Work}
% player control over procedural content generation
Better player modelling would allow more accurate reflection of user actions and intentions. The closely related field of dynamic difficulty adjustment already uses a number of techniques for more accurate and efficient measures of player skill, and many of these techniques may be transferable to modelling other player aspects.

\section{Conclusions}
% a comparison and evaluation of approaches, and an indication of the outstanding, unsolved, issues and problems.
Though PCG is a wide and in many senses well-established field, with increasing adoption by industry, there are still a number of outstanding issues that hinder effective research in this area. The diversity of approaches to common problems has led to no standard, portable methods for procedural generation of any single content type, and this lack of standardisation has an effect on many areas of the field. Without a standard approach, research focuses have been divided between a number of largely-incompatible alternative techniques, though search-based procedural content generation techniques appear to be increasingly popular.

Outside the topic itself, procedural content generation techniques can be applied to generate content from a number of closely related fields, and there is scope for further research in many areas, from procedurally-specified particle systems to procedurally-varied squad AI. Of particular interest is the overlap between existing dymanic difficulty adjustment techniques and the area of player-adaptive PCG, as this is currently an under-researched area with scope for novel techniques for automatic personalisation of generated content.

In conclusion, this paper has exhibited and classified a wide variety of procedural content generation techniques, and provided commentary on common aspects and transferable approaches. Some areas are just beginning to adopt standard methods for the generation of certain types of content (component-assembly method, search-based procedural content generation), while other areas (player-adaptive procedural content generation) still have a diverse variety of approaches. In all cases artists, designers and users alike stand to benefit from the further development of procedural content generation techniques.

% \footnote{General Purpose Procedural Content Generators seem even further off, though Procedural Procedural Content Generator Generators exist already}
% The modern computer games industry makes use of a wide variety of procedural content generation (PCG) techniques that serve a number of purposes during both the development and execution of a game. This paper provides a high-level classification of a range of techniques used for different types of content, and attempts to discern common aspects and transferable approaches that can help to promote a more unified, standard approach to content generation for games. 

% We can see that a variety of <stakeholders> benefit from the improvement of procedural content techniques, and that there are a wide range of existing techniques used for a plethora of different reasons. 
% Nitwit Blubber Oddment Tweak

%ACKNOWLEDGMENTS are optional
% \section{Acknowledgments}
% \newpage
% \section{dummy}
\newpage
\bibliographystyle{abbrv}
\bibliography{IRP}
%\balancecolumns
\balancecolumns
% That's all folks!
\end{document}

You should read and summarise these articles, producing a 8 page, using a two-column format, survey article indicating the background to the problem, the methods and results presented in your group of articles, a comparison and evaluation of approaches, and an indication of the outstanding, unsolved, issues and problems.
