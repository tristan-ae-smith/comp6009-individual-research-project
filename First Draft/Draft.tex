\documentclass{./acm_proc_article-sp}

\begin{document}

\title{Procedural Content Generation for Computer Games}
\subtitle{A survey of techniques used for procedural content generation for computergames, classified by beneficiary.}

\numberofauthors{1}
\author{
\alignauthor
Thomas Smith\\
       \affaddr{Electronics and Computer Science}\\
       \affaddr{University of Southampton}\\
       \email{taes1g09@ecs.soton.ac.uk}
}

\maketitle
\begin{abstract}
Modern computer games make use of a wide variety of procedural content generation (PCG) techniques that serve a number of purposes during the development of a game. 
Improve variety, allow developers to populate large areas, allow customisation to player preferences, reduce assets, allow control/tweaking, reduce labor costs
Automating content generation, or augmenting manual content production


% Hundreds of millions of people play computer games every day. For them, game content–from 3D objects to abstract puzzles– plays a major entertainment role. Manual labor has so far ensured that the quality and quantity of game content matched the demands of the playing community, but is facing new scalability challenges due to the exponential growth over the last decade of both the gamer population and the production costs. Procedural Content Generation for Games (PCG-G) may address these challenges by automating, or aiding in, game content generation. PCG-G is difficult, since the generator has to create the content, satisfy constraints imposed by the artist, and return interesting instances for gamers. Despite a large body of research focusing on PCG-G, particularly over the past decade, ours is the first comprehensive survey of the field of PCG-G. We first introduce a comprehensive, six-layered taxonomy of game content: bits, space, systems, scenarios, design, and derived. Second, we survey the methods used across the whole field of PCG-G from a large research body. Third, we map PCG-G methods to game content layers; it turns out that many of the methods used to generate game content from one layer can be used to generate content from another. We also survey the use of methods in practice, that is, in commercial or prototype games. Fourth and last, we discuss several directions for future research in PCG-G, which we believe deserve close attention in the near future.
\end{abstract}

% A category with the (minimum) three required fields
% \category{H.4}{Information Systems Applications}{Miscellaneous}
%A category including the fourth, optional field follows...
% \category{D.2.8}{Software Engineering}{Metrics}[complexity measures, performance measures]

% \terms{Theory}

% \keywords{ACM proceedings, \LaTeX, text tagging} % NOT required for Proceedings

\section{Introduction}
Intoduction to the topic, explanation of lack of structure. Reference to the age of things, Elite\cite{elite}, nethack, increasing use in successful commercial games. The need for procedural content generation.

\section{Background}
Building on the work of Hendrikx et. al \cite{hendrikx2012procedural} this paper provides a modern overview of the range of procedural content generation techniques used in games today.
\subsection{Methodology}
Given the broad range of procedural content techniques, this review organises approaches first by the primary user of each method. Lists a number of the main purposes that PCG is used for within that category, along with examples and academic literature in that area.


\section{Artists}
%methods and results
When producing the raw content that goes into a game, procedural generation techniques can provide a more efficient method or greater variety than building everything by hand. Many of these techniques are not unique to the field of computer games and are also used to produce computer-generated graphics for all kinds of media, from animated movies to photorealistic bcakgrounds in advertisements and print media.

\subsubsection{Textures}

Perlin Noise
Grammars
Long history of using perlin noise effects for material textures. Originally developed by Ken Perlin in <year>, it provides <definition>. Used for everything from clouds to marble.
Pattern-based textures. Ideal for providing high-resolution texturing across large landscapes. Algorithmically generates textures with specified features, useful for either repetetive applications such as hich-rise city blocks or aperiodic such as natural landscapes \cite{patternTextures}
\subsubsection{Models}
visual variety, procedural construction
fill space - speedtree\cite{speedtree}, procedural cities
Borderland's guns
Grammars
Speedtree
\subsubsection{Animation}
Rather than create animations for all possibilites (counterexample - assassins creed 3 had <number> of distinct animations)
Respond to conditions that weren't known at design time: user content generation - Spore\cite{Spore}, allows charaters to react to a vast range of physical conditions - Jedi Unleashed force push, Emotion engine.
\subsubsection{Effects}
Procedural generation of environmental effects such as fire, water, smoke and clouds. Provides believable variety (starter point for citations in \cite{hendrikx2012procedural})
Particles
GAR?
Procedural rendering effects - allows graphical styles that are radically distinct from traditional photorealistic techniques \cite{kowalski1999art}
\subsubsection{Music}
Allows great variety - specify a style, and get infinite variations. \cite{collins2009introduction} <Does this belong in designers? often reacts to in-games event, used to build mood>

%comparison and evaluation of approaches
\subsection{Benefits}
Download sizes - procedural variation - Borderland's enemies
\subsubsection{Future work}

\section{Designers}
%methods and results

\subsection{Varieties}
\subsubsection{Content scale}
Speedtree
\subsubsection{Replayability}
\subsubsection{Challenge}

%comparison and evaluation of approaches
\subsection{Benefits}
\subsubsection{Future work}

\section{Users}
%methods and results

\subsubsection{Experience}
Valve's AI Director
Bethesda's Radiant Storytelling
\subsubsection{Agency}
Typically, no direct player control over adaptive generation (cite hamlet)
however, in some games that make use of procedural generation it can be a benefit to give the player some degree of direct control over the generation process - for example a recent addition to the GAR was the `'Weapons Lab', a portion of the game where players may spend in-game resources to customise their procedurally-generated weapons. (cite GAR's weapons lab)
Radiant Story?
GAR's weapons lab

%comparison and evaluation of approaches
\subsection{Benefits}
\subsubsection{Future work}
player control over procedural content generation

\section{Conclusion}
% a comparison and evaluation of approaches, and an indication of the outstanding, unsolved, issues and problems.
We can see that a variety of <stakeholders> benefit from the improvement of procedural content techniques, and that there are a wide range of existing techniques used for a plethora of different reasons.

%ACKNOWLEDGMENTS are optional
% \section{Acknowledgments}

\bibliographystyle{abbrv}
\bibliography{draft}
%\balancecolumns
\appendix
\section{Project Brief}
% \documentclass[12pt]{article}
%\usepackage{a4wide}
%\usepackage[small,compact]{titlesec}
\usepackage[parfill]{parskip}
\date{}
%\raggedbottom
\thispagestyle{empty}
\setlength{\topmargin}{-0.2in}
\setlength{\textheight}{8.7in} 
\begin{document}
\begin{center}
{\LARGE Procedural Content Generation\\[0.25cm] for Computer Games}


Thomas Smith (\texttt{taes1g09@ecs.soton.ac.uk})\linebreak
{\bf Supervisor:} Rikki Prince (\texttt{rfp@ecs.soton.ac.uk})
\end{center}

{\bf Problem:} Procedural Content Generation (PCG) is the name applied to the growing field of techniques that attempt to provide content for modern games through the use of algorithmic generators rather than manual construction. As AAA games get larger and indie studios get smaller, all varieties of computer game developer have an incentive to automate the production of material of all kinds for their games. A large number of specialised techniques exist, each tailored to generating a particular kind of content, from rulesets to textures to enemy or object behaviour. Increasingly, concepts from the field of AI are being used to improve the quality and relevance of the generated content, whether that be through intelligently breeding possibilities from a population of solutions or applying machine learning to create a map between player behaviour and optimal content. Each genre of game and each kind of generated content has a range of potential techniques that may be useful for its production, and typically the PCG techniques used in games are individual bespoke applications of these various methods as each game requires different content and there are not yet standard approaches to integrating multiple kinds of content generation.


%In modern computer game development, content production accounts for a large proportion of the initial (and in some cases, ongoing) outlay. As both budgets and in-game worlds get larger, there is increasing demand to offload some of these production efforts to automated systems. The concept of procedural content generation (PCG) has been around for some time, and it has been used in many successful games. When content is generated manually or algorithmically during the design phase of a game, it can only be created according to the designer's expectations of the players' needs. By instead generating content during the execution of the game, and using information about the player(s) as one of the system's inputs, PCG systems should be able to produce more dynamic experiences that can be far more tailored to enhance individual player's experiences than anything manually created.

{\bf Goals:} Many existing papers on PCG deal with a single aspect or particular technique. A smaller number consider a range of previous techniques for the generation of a particular range of game content, such as terrain or cities, and draw comparisons on their relative merits. The goal of this project will be to attempt a high-level classification of a large range of techniques used for many different types of content, and attempt to discern common aspects that can help to promote a more unified, standard approach to content generation for games. The final output will consist of a detailed report, an A1 poster and accompanying presentation, and hopefully a short paper for submission to the Fourth Workshop on Procedural Content Generation in Games at FDG2013.

{\bf Scope:} In order to attempt to ensure that the project goals remain achievable, the scope should be restricted to only recognised and established PCG techniques with obvious application to games. It may be useful to consider related areas for comparison, such as player-driven content generation or procedural PCG generators, but the focus should remain limited to applicable and useful methods in order to facilitate classification and comparison.

\end{document}

lots of genres
lots of genre-specific solutions
range of content - levels, music, textures, rules, weapons, storyline? meta
range of methods

goals - necessarily highlevel %look up pdfpages
\balancecolumns
% That's all folks!
\end{document}

You should read and summarise these articles, producing a 8 page, using a two-column format, survey article indicating the background to the problem, the methods and results presented in your group of articles, a comparison and evaluation of approaches, and an indication of the outstanding, unsolved, issues and problems.
